\chapter{Discrete-Event Simulations (DES)}

Discrete Event Simulation is a generalization of the Simulation we have seen so far. Consider the Stochastic Simulations (e.g. Gillespie's algorithm and PRISM), their limitations lies if the events of a system could be not instantaneous. Considers the case when the frequency of events follow different distributions (e.g. fixed delay, uniform distribution within an interval, ...), in that case we need to generalize the simulation to be able to allow these timing issues in our system. \par

Discrete Event Simulation (DES) does not rely on the memory-less property of Stochastic Simulations, \textbf{we have to keep track of what is going on in our system}. DES implement an \textbf{event list} to be aware of the time passing. In an event list we have:

\begin{itemize}
    \item \textbf{State:} is a description of system configuration in terms of a set of variables.
    \item \textbf{Activity:} is a process that involves a sequence of updates of the state variables (events) over time. It has a \textbf{duration}.
    \item \textbf{Event:} is an instantaneous update of the state variables. They can correspond to the start and end of an activity.
    \item \textbf{Event notice:} is a description of a future event with the time at which it will happen.
    \item \textbf{Future Event List:} is a list of event notices ordered by time.
\end{itemize}

The system will be consisting in a number of activities that runs concurrently. Each of these activities that are running and that can schedule events in the future. Each activity can have variables describing the state of it and all the variables of all activities describe the state of the simulation.\par
An activity can be started at the beginning of the simulation or can start after a delay. Each activity as a duration (which is a different concept that delay) which is the time in between the start event and end event. Obviously the start event describe the beginning of the execution of that activity, while the end event describe the termination. An activity can have other events.\par
The behavior of the activity can be specified by some code which describe the computation done by the activity and schedule some events of the activity

\section{Future Event List}
the Future Event List (FEL) is the queue we have keep track of the future events scheduled. Each element of the list is a pair $(t_{i}, Event_{i})$ where:
\begin{itemize}
    \item $t_{i}$ is a time stamp.
    \item $Event_{i}$ is a description of the event.
\end{itemize}

It is ordered by increasing time of the time stamps.\par

%immagine lezione #14 in pag 11/19%

\subsubsection{Generalization of the simulation algorithm using the FEL}
\begin{itemize}
    \item initialize state variables, the FEL and a \textbf{global clock variable} $T = 0$. At the beginning the FEL contains only the starting events of each activities in our system.
    \item at each step:
        \begin{itemize}
            \item remove the first event notice $(t, Event)$.
            \item handle Event. This could cause the update of state variables, adding one or more event notices to the FEL and removing one or more scheduled events from the FEL.
            \item update the global clock $T = t$
        \end{itemize}

    \item we stop when the clock reach a stop time $T_{stop}$ s.t. $T = T_{stop}$ or the FEL is empty. 
\end{itemize}

\subsection{Conditional and Primary Events}
To handle events that have to wait for a condition in ordered to be enable we need to divide events in two types:

\begin{itemize}
    \item \textbf{Primary events:} events whose occurrence in scheduled at a certain time.
    \item \textbf{Conditional events:} events that are triggered by a certain condition becoming true. \textbf{Conditional events are untimed}. For this reason we can store Conditional events at the beginning of the FEL or in an apposite data structure.
\end{itemize}

\subsection{Revised Simulation Algorithm}
We rewrite the simulation algorithm taking into account the two types of events possible:
\begin{itemize}
    \item Initialize variables, FEL and T
    \item Iterate:
        \begin{itemize}
            \item If there is a conditional event enabled, remove and process it (To does not change), otherwise remove and process the first primary event. Finally update T
        \end{itemize}
    \item Continue until the global clock reaches the time stop $T = T_{stop}$ or the Future Event List is empty 
\end{itemize}

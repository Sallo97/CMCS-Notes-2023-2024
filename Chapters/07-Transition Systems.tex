\chapter{Transition Systems}
We want to start considering models that permit us to \textbf{study the global behaviour of a system}. Assume we are interested in studying whether a system can reach a bad state:
\begin{itemize}
    \item If we use ODEs we can state only the "average" behaviour of a system.
    \item if we us a Stochastic simulator we can only study a number of different possible behaviours.
\end{itemize}
Both approaches does not guarantee that the system will never reach a bad state. To solve this problem we need to introduce a new way of modeling the system behaviour by using Transition Systems.\par
Transition Systems are \textbf{another model of behavior}, so another way to specify how a value changes over time.

\section{What is a Transition System}
A Transition System is a pair $(S, \rightarrow)$ where:
\begin{itemize}
    \item S is a \textbf{set of states}.
    \item $\rightarrow \subseteq S \times S$ is the \textbf{transition relation}.
\end{itemize}
Given two states $s, s^{'} \in S, (s, s^') \in \rightarrow$, then we can write $s \rightarrow s^'$.\par 
If instead we have $s \nrightarrow$ it denotes that there exists no $s^' \in S$ s.t. $s \rightarrow s^'$.\par
Note that:

\begin{itemize}
    \item The set of states can be infinity (typically is assumed to be recursively enumerable).
    \item Transitions describe system state changes.
    \item A state may have more than one outgoing transitions ( $s \rightarrow s^'$ and $s \rightarrow s^{''}$ capturing \textbf{non-deterministic behaviors}. The fact that $s_1$ may evolve in either $s_2$ or $s_3$ does not necessary mean that there is a random choice between the two possibilities. The state choosen could depend froma timer, a scheduler, a probabilistic choice and so on. In general, non-determinis is an \textbf{abstraction} of choice criterion that we simply do not want to model.
\end{itemize}

\section{Trace}
A trace $t$ in a Transition System is a path, meaning a sequence of states t = $s_0, s_1, ..., s_i, s_{i+1}, ...$ such that for each $s_i$ and $s_{i+1}$ ($i \in from the initial state \mathbb{N}$ it holds $s_i \rightarrow s_{i+1}$. $s_0$ is always chose as the \textbf{initial state}.
Note:
\begin{itemize}
    \item t = $s_0$ is the \textbf{minimal trace}.
    \item a trace t is maximal if either t is infinite or $t = s_0, s_1, ..., s_n$ and $s_n \nrightarrow$.
\end{itemize}

\section{Reachability}
Given a state s in a Transition System $(S, \rightarrow)$ we can say that s is reachable if starting from the initial state $s_0$ there exists a trace that starts from $s_0$ and ends at $s$: $t = s_0, s_1, ..., s_n, s$.\par
Very often a Transition System is used to verify that in the model we can reach a particular (good or bad) state.

\par{Class of Transition Systems - Kripke Structures}
A Kirpke Structure is a class of Transion Systems where states are characterized by a set of \textbf{atomic propositions} that can either be true or false. \par
Given a (finite) state of atomic propositions $AP$, a Kripke Structure $K$ is a Transition System $(S, \rightarrow)$ where $S = \mathcal{P}(AP)$.
\begin{itemize}
    \item $\mathcal{P}(AP)$ denotes the powerset of AP.
    \item The interpretation is that an atomic proposition $a$ is contained in a state if and only if it is true in that state
\end{itemize}

\textbf{Note: a state $s$ contains the proposition that are true when inside it.}

\section{Transition Systems over a set of variables}
Given a set of variables $X = \{ X_1, X_2, ..., X_n\}$ and a set of domains $\{D_1, D_2, ..., D_n\}$ s.t. $D_i$ is the domain of $X_i$, a Transition System over X is a Transition System $(S, \rightarrow)$ where $S = D_1 \times D_2 \times ... \times D_n$.

\subsection{Transition Rules}
Given a Transition System over a set of variables, it can be specified b y giving a set of \textbf{transition rules} having the following form:
\begin{center}
    guard $\rightarrow$ update
\end{center}
where:
\begin{itemize}
    \item \textbf{guard}: is \textbf{conjunction of conditions} on the state variables, each having the form $X_i$ op Exp (op is a comparison operator).
    \item \textbf{update}: is a \textbf{conjunction of assignments} to state variables, each having the form $X^{'}_{i} = $ Exp, with $X^{'}_{i}$ denoting the new values of $X_i$ 
\end{itemize}

The idea is that the transition relation contain a transition between each pair of states $s_1, s_2$ s.t.:
\begin{itemize}
    \item $s_1$ satisfies the guard.
    \item  $s_2$ can be obtained by applying to $s_1$ the assignments described in update.
\end{itemize}

%starts Lecture 8%

\section{Labeled Transition Systems}
Consider a Concurrent Interaction Systems where we have several components that are in some extent independent, but they can interact with each other. Representing the system as a whole would be overwhelming, requiring lots of variables and rules. Instead it would be more natural to represent it in a \textbf{Compositional Way} by separating the variables and rules depending of which component they refer to (similar approach to object oriented programming where we create programs that use several classes, but each one is defined separately).\par
In order to define this compositionality feature, we need to find a way to define a transition system based on the individual components, \textbf{taking into account that they could interact between each other}. This means that in the case of two components that interact with each other, when merging their transition systems I should be able to unify their rule that explain the interaction between them. In order to recognize these activities we introduce a \textbf{labels} and define Labeled Transition Systems.\par
Labeled Transition Systems (LTS) are an extended version of Transition Systems in which transitions are enriched with labels. Formally, a LTS is a triple $(S, L, \rightarrow)$ where:
\begin{itemize}
    \item S is a set of \textbf{states}
    \item L is a set of \textbf{labels}
    \item $\rightarrow \subseteq S \times L \times S$ is a \textbf{labeled transition relation}. the triple $(s, \ell, s^{'}) \in L$ is usually denoted as $s \xrightarrow{\ell} s^{'}$. 
\end{itemize}

\subsection{Transition Labels}
Transitions labels $\ell$ describe the action performed by the system (or a single components) during a transition. We distinguish two types of labels:

\begin{itemize}
    \item Labels that use $\tau$ describe an internal action, that is one that is performed in isolation by the single component that contains it.
    \item Other labels ($a, b, c, ...)$ describe potential actions the system (or component) could perform by interaction with some other component.
\end{itemize}

\subsection{Synchronization}
There are two ways to model synchronization:
\begin{itemize}
    \item \textbf{Binary synchronization:} it specify the interaction between two components. A transition with label $a$ of one components has to be performed together with a transition with label $\overline{a}$ (same symbol, but overlined) of another component.

    \item \textbf{Global synchronization:} actions that are synchronized among all system component. All components having a transition with the same label $a$ must perform such a transition together.
\end{itemize}

\textbf{Note:} the synchronization of a number of transitions result in a new $\tau$ transition.
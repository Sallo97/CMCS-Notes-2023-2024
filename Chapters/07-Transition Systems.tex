\chapter{Transition Systems}
We want to start considering models that permit us to \textbf{study the global behaviour of a system}. Assume we are interested in studying whether a system can reach a bad state:
\begin{itemize}
    \item If we use ODEs we can state only the "average" behaviour of a system.
    \item if we us a Stochastic simulator we can only study a number of different possible behaviours.
\end{itemize}
Both approaches does not guarantee that the system will never reach a bad state. To solve this problem we need to introduce a new way of modeling the system behaviour by using Transition Systems.\par
Transition Systems are \textbf{another model of behavior}, so another way to specify how a value changes over time.

\section{What is a Transition System}
A Transition System is a pair $(S, \rightarrow)$ where:
\begin{itemize}
    \item S is a \textbf{set of states}.
    \item $\rightarrow \subseteq S \times S$ is the \textbf{transition relation}.
\end{itemize}
Given two states $s, s^{'} \in S, (s, s^') \in \rightarrow$, then we can write $s \rightarrow s^'$.\par 
If instead we have $s \nrightarrow$ it denotes that there exists no $s^' \in S$ s.t. $s \rightarrow s^'$.\par
Note that:

\begin{itemize}
    \item The set of states can be infinity (typically is assumed to be recursively enumerable).
    \item Transitions describe system state changes.
    \item A state may have more than one outgoing transitions ( $s \rightarrow s^'$ and $s \rightarrow s^{''}$ capturing \textbf{non-deterministic behaviors}. The fact that $s_1$ may evolve in either $s_2$ or $s_3$ does not necessary mean that there is a random choice between the two possibilities. The state choosen could depend froma timer, a scheduler, a probabilistic choice and so on. In general, non-determinis is an \textbf{abstraction} of choice criterion that we simply do not want to model.
\end{itemize}

\section{Trace}
A trace $t$ in a Transition System is a path, meaning a sequence of states t = $s_0, s_1, ..., s_i, s_{i+1}, ...$ such that for each $s_i$ and $s_{i+1}$ ($i \in from the initial state \mathbb{N}$ it holds $s_i \rightarrow s_{i+1}$. $s_0$ is always chose as the \textbf{initial state}.
Note:
\begin{itemize}
    \item t = $s_0$ is the \textbf{minimal trace}.
    \item a trace t is maximal if either t is infinite or $t = s_0, s_1, ..., s_n$ and $s_n \nrightarrow$.
\end{itemize}

\section{Reachability}
Given a state s in a Transition System $(S, \rightarrow)$ we can say that s is reachable if starting from the initial state $s_0$ there exists a trace that starts from $s_0$ and ends at $s$: $t = s_0, s_1, ..., s_n, s$.\par
Very often a Transition System is used to verify that in the model we can reach a particular (good or bad) state.

\par{Class of Transition Systems - Kripke Structures}
A Kirpke Structure is a class of Transion Systems where states are characterized by a set of \textbf{atomic propositions} that can either be true or false. \par
Given a (finite) state of atomic propositions $AP$, a Kripke Structure $K$ is a Transition System $(S, \rightarrow)$ where $S = \mathcal{P}(AP)$.
\begin{itemize}
    \item $\mathcal{P}(AP)$ denotes the powerset of AP.
    \item The interpretation is that an atomic proposition $a$ is contained in a state if and only if it is true in that state
\end{itemize}

\textbf{Note: a state $s$ contains the proposition that are true when inside it.}

\section{Transition Systems over a set of variables}
Given a set of variables $X = \{ X_1, X_2, ..., X_n\}$ and a set of domains $\{D_1, D_2, ..., D_n\}$ s.t. $D_i$ is the domain of $X_i$, a Transition System over X is a Transition System $(S, \rightarrow)$ where $S = D_1 \times D_2 \times ... \times D_n$.

\subsection{Transition Rules}
Given a Transition System over a set of variables, it can be specified b y giving a set of \textbf{transition rules} having the following form:
\begin{center}
    guard $\rightarrow$ update
\end{center}
where:
\begin{itemize}
    \item \textbf{guard}: is \textbf{conjunction of conditions} on the state variables, each having the form $X_i$ op Exp (op is a comparison operator).
    \item \textbf{update}: is a \textbf{conjunction of assignments} to state variables, each having the form $X^{'}_{i} = $ Exp, with $X^{'}_{i}$ denoting the new values of $X_i$ 
\end{itemize}

The idea is that the transition relation contain a transition between each pair of states $s_1, s_2$ s.t.:
\begin{itemize}
    \item $s_1$ satisfies the guard.
    \item  $s_2$ can be obtained by applying to $s_1$ the assignments described in update.
\end{itemize}
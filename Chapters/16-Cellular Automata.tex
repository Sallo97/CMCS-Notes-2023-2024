\chapter{Cellular Automata}
Agent-Based Modeling is a modeling approach in which system components are represented as agents able to take decisions, perform actions, and interact with other agents or/and the environment.\par
Usually the behaviors of an agent are specified using a high-level programming language.\par
Agent-Based Simulation is a form of Discrete Event Simulation that consists in "executing" agents concurrently.

\section{Considering the environment}
Agents can move in a 2D or 3D environment. Agent position and spatial charateristics of the environment will influence the system dynamics (e.g. agents can only interaction with neighbours; there could be obstacles; resources are distributed across the environment; areas could have different charateristics).

\section{Cellular Automata}
Cellular Automata (CA) allow to describe environments in 1D, 2D or 3D. The environment is described as a matrix of cells. Each cell is defined by its own state that can change by means of rules. Cellular Automata can be used to model Complex Systems with spacial structure.\par

\subsubsection{Multicellularity}
Cellular Automata uses as reference multicellularity ad a way to build complex living systems.\par
Multicellular systems are composed by many copies of cell, our unique fundamental unit. Cells will interact with the ones in his proximity, influencing the fate and behavior of each cell. The result is an heterogeneous system composed by differentiated cells that act as specialized units, even if they all contain the same genetic material and have essentially the same structure.

\section{Modelling cellular systems}
